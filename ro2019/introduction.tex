
\section{Introduction}

Publicly funded efforts around the world currently are underway to 
	ensure that computational components of scientific research
	can be made ``reproducible'' or ``replicable''.
The ongoing discourse about the perceived ``reproducibility crisis in science''~\cite{fanelli_opinion:_2018} 
	is just one illustration of the importance of these efforts.
The energy invested in the wide-ranging debate over the precise meanings of the 
	terms \emph{reproducible}, \emph{replicable},
        \emph{transparent}, etc.\
        \cite{drummond2009replicability,carolegoble2016what,freire2016reproducibilitya,goodman2016what,ioannidis2017reproducibility,herouxtoward,plesser2018reproducibility,barba2018terminologies,committeeonreproducibilityandreplicabilityinscience2019reproducibility}, with 
	respect to research results, processes, and settings is perhaps an even greater indication of 
	both the significance of these efforts and the challenges they face.
For while each effort aimed at facilitating reproducible computing in the
	sciences must clearly define its mission and apply the bulk of its resources
	to the specific problems it sets out to address, these efforts necessarily do
	so within the context of broader discussions about the nature, importance,
	and precise definitions of the qualities of science we wish to extend to computing
	over the longer time scale.

Within a particular effort it is useful to define terms such as \emph{reproducible} operationally.
For example, in the Whole Tale project \cite{WT2019,brinckman2019computing} we define a \emph{Reproducible Tale} as one 
	that \emph{includes sufficient information for the Tale to be re-executed for the review 
	and verification of results}.
Adopting this definition allows us to focus our requirements analysis, system design,
	and software implementation efforts on the specific problems Whole Tale is funded to solve
	and the use cases we aim to support.
Supporting publishers who request authors to include all new data, 
	code, and workflows needed to reproduce computed artifacts supporting
	claims in a paper is one such use case targeted by Whole Tale.
We anticipate that facilitating re-execution of code used to generate
	key products of a study will enable publishers routinely to confirm that
	provided data and code do in fact produce those results---thereby addressing
	a key dimension of the reproducibility challenge currently facing science.

At the same time, it is critical that efforts like Whole Tale contribute to a global 
	vision of computational reproducibility in the sciences, and clearly situate 
	its particular mission, use cases, and engineering deliverables in this context.
For while the particular technical problems that Whole Tale and similar projects
	aim to address are particularly pressing, current efforts by no means
	represent the entire landscape of concepts, problems, and technical options
	that will require further discussion, clarification, and analysis if we are to meet
	the challenges of reproducibility.
In particular, current engineering efforts are unlikely to elevate the computational components
	of research to the level of reproducibility historically expected of studies in the
	pure natural sciences such as physics, chemistry, and biology.

Consequently, we view the Whole Tale project---as currently chartered and funded---as just a step
	towards the kind of platforms, infrastructure, and standards
	needed to enable researchers using computing technology to routinely 
	achieve the reproducibility long considered the essence of science as a whole.
In support of this longer-term vision, we outline in this paper % just
 a few of the issues	we aim to investigate and discuss with the broader community over the next few years.
We anticipate that future iterations of Whole Tale and its sibling efforts 
	will be driven in part by the problem definitions and solution proposals we collectively
	develop between now and then.

In the remainder of this paper 
we briefly discuss four topics we plan to investigate
	in the course of the Whole Tale project.
In Section\,\ref{sec-reproducibility} we review the general notion of
reproducibility in science, and in Section\,\ref{sec-repeatability} 
	highlight how digital computing in principle makes possible a completely new kind of reproducibility: 
	\emph{exact repeatability}. 
We emphasize that the notion of \emph{transparency}---long a critical element of
	reproducibillity in the pure natural sciences---has a role to play even for those computational components
	of research where exact repeatability is feasible.
In Section\,\ref{sec-terminology} we provide an overview of several dimensions of the terminological debate around reproducibility
	generally, and propose that a pluralistic approach to defining key terms is essential if a general 
	concept of reproducibility is to be shared across disciplines.
In Section\,\ref{sec-limitations} we summarize a number of limitations on exact repeatability in practice, and in Section\,\ref{sec-transparency}
	show how science-oriented provenance queries can mitigate such limitations by maintaining
 	the transparency most essential to reproducibility in science.
Throughout, we highlight the role that Research Objects \cite{bechhofer2013whya}
	can serve in supporting and maintaining reproducibility 
	by encapsulating the information needed to rerun the computational steps in a study, 
	by disambiguating claims about reproducibility, 
	and by enabling transparency via queries of provenance information packaged in the object. 



%%% Local Variables: 
%%% mode: latex
%%% TeX-master: "main"
%%% End: 
